\documentclass{article}
\usepackage{graphicx} % Required for inserting images
\usepackage{amsmath}
\usepackage{url}
\usepackage{indentfirst}
\usepackage[margin=1.5in]{geometry} % Sets default margins to 1 inch

\title{Blackjack Strategy Analysis}
\author{Suraj Bhardwaj}
\date{May 2024}

\begin{document}

\maketitle
\pagebreak
\section{Background}
The game of blackjack can be boiled down into a very simple set of rules. The goal of the player is to have highest value hand on the table without its value exceeding 21. Each card in blackjack is assigned a certain value. The value of all numbered cards is the value of the number itself and any face card holds a value of 10. Aces holds a value of either 1 or 11 depending on what forms a stronger hand. In any hand, if there is an ace whose value is set to 11 (making the strongest hand) then that hand is referred to by its value and “soft” (example: soft 15 consists of an ace and a 4). Any hand which does not include an ace or where all aces have a value of 1 are referred to as “hard”. The game starts with a dealer who deals out 2 card hands to all players including himself. All cards except for one of the dealer’s cards are facing up, viewable by all players. The dealer then iterates through all participants letting them play out their turns to completion before playing his own turn. \\

Each player has only a few moves available to them. The first option is to "stand". If one stands, they are deciding that they would like to end their turn, after which, the next player then participates. The next option is to "hit". To hit means to receive an additional card from the dealer. The dealer will then deal out another card to the player and the player can continue their turn. The next option that a player can take advantage of, is to "double down" or "double". To double means to receive one additional card and double one’s betting amount (and thus their potential returns) simultaneously. Doubling terminates the player's turn. The final option is to "split", the player has the option to split only at the beginning of the game, immediately after being dealt their initial 2 card hand considering that they have 2 of the same cards (same value or face). To split means to divide that 2-card hand into 2 separate 2 card hands doubling the total betting amount but keeping the same bet amount per hand. The original two cards are split, and each new hand is dealt another card. \\

After the dealer has dealt out all the hands but before the players have played out their turns, the dealer checks their upside-down card to check whether he is holding a blackjack. If he is holding a blackjack (hand that equals 21 in value) with his 2-card hand, all players lose unless they too are holding a blackjack with their initial 2 card hands. After the players have played out their turns, the dealer sweeps away all the bets from the players whose hands exceed 21. The dealer then flips his second card over and plays out his turn. \\

The dealer in blackjack plays under certain constraints. Each casino has their own rules for how the dealer must play however the most popular and common way is that the dealer hits until their hand’s value exceeds 17 (counting any ace’s as 1). The dealer does not have the option to double (because they are not betting anything) or split. Each player, when playing must place a minimumbet or higher to be dealt in. When the player wins the dealer returns them twice their initial bet (100\% return).

\section{Simulation Details}

For the simulation that I built, certain rules had to be defined. I have tried my best to adhere to the most popular/standard casino rules while also including rules which give the casino an advantage to use the findings of this project as a minimum baseline for potential returns. The simulated game plays with 8 decks (however this can be changed) with a randomized shoe and an average penetration of 7 decks. This simulation pays 3:2 for player blackjacks and allows the player to double after splitting. The player is allowed to double until they have 4 hands maximum in this simulation regardless of the card being split (some casinos have certain rules regarding splitting aces). To add on, some casinos offer the option to “surrender” where the player would be able to take back half their bet before any player plays out their turn, after the dealer deals out all the players' cards. This option is not offered at all casinos and is not implemented in my simulation of the game. Lastly, the player is presented with the option to take “insurance”, at some casinos, where they make a side bet of a minimum of half their bet on whether the dealer is holding blackjack in the two cards they were initially dealt. This side bet pays out 2:1. This option is also not implemented in my simulation of blackjack.

\section{How Casinos Increase Their Advantage}
Casinos already hold a significant advantage in blackjack since the dealer plays last so any players that bust automatically lose regardless of whether the dealer busts or not. Additionally, should the dealer deal himself a blackjack, they win immediately without giving the player the opportunity to reach 21. To add on, casinos do not exhaust the entire stock of cards before reshuffling and restarting. Once all the cards have been shuffled, the dealer will place a cut card towards the bottom of the cards separating approximately one deck from the rest. The cards will then be dealt until that card has been hit. Once that card comes out, the current hand will be finished and then all the cards will be shuffled, and the process starts anew. Blackjack is often played with either 6 or 8 decks (very few casinos play with 2 or 4 decks but some do) which varies the composition of the shoe (all cards above the cut card).

\section{Strategies}
There are 4 strategies of play which are analyzed in this simulation: dealer strategy, basic strategy, high-low card counting, and high-low card counting with bet spreads. The dealer strategy was discussed previously. Basic strategy is statistically the best way to play blackjack, assuming that the dealing of each card is an independent event. Card counting is the best way to play blackjack, based on the knowledge that the dealing of each card is not an independent event with a set probability but rather an event dependent on the prior dealings. High-Low card counting is the most widely used method of counting cards. This method splits all the cards up into 3 categories, high, low, and medium. The low cards are, 2, 3, 4, 5, and 6. 7, 8, 9 are medium cards. And 10, Jack, Queen, King, and Ace are all high cards. As the game is played and cards are dealt, a running count of the cards is kept. When a low card comes out, 1 is added to the count. When a high card is dealt, 1 is taken from the count. When a medium card comes out, nothing is added or subtracted from the count. This running count along with the number of decks remaining in the shoe is used to derive the "true count". A higher true count is beneficial to the player and means that the probability of a high card being dealt is higher. \\

Research into the best strategy in blackjack has led us to basic strategy and established the principles of card counting. There are some other strategies however they will not be discussed in this paper. \\

The idea behind high-low card counting is that when there are more higher cards in the deck, waiting to come out, the player holds an advantage because they are allowed to stand on any number whereas the dealer can only stand over 16. For example, if we know that the probability of a 10 coming out of the deck is high, and the value of our hand is over 11. It would be strategic to stand because if a ten were to come out, we would bust. As such, the player would choose their move according to the count such that they use the probabilities to their advantage.

\section{Research}
The developed program simulates 1,000,000 hands of blackjack using an 8-deck shoe with variable deck penetration. Beyond simply tallying the aggregate wins, losses, and draws based on a 1-unit bet, the software captures a comprehensive dataset for analysis. Specifically, it records the cumulative bankroll progression throughout the simulation, the outcome of every individual hand, and the floored 'True Count' calculated prior to each deal. A secondary program utilizes this JSON output to generate the visualizations provided in this report. Prior research points to the fact that basic strategy Is the optimal method without counting cards while counting cards is the optimal strategy overall.

\section{Results}
The data generated by the simulations reveals a distinct hierarchy in strategic performance, substantiating the claim that card counting turns the statistical advantage in favor of the player. Interestingly, a comparison of raw "Win Percentages" yields a counter-intuitive result: the Dealer Strategy maintained the highest win rate at 45.91\%, followed by the Card Counting variations at 43.77\%, and finally Basic Strategy at 42.83\%. However, win rate is a poor metric for success in Blackjack due to the payout structure of 3:2 for blackjacks and the ability to double down. \\

The true efficacy of the strategies is observed in the "Final Results" (cumulative bankroll) and the "House Edge." The Dealer Strategy resulted in a catastrophic loss of 79,451 betting units with a calculated House Edge of 7.95\%. Basic Strategy significantly reduced losses, finishing down 12,450.5 units with a House Edge of 1.25\%, proving it is far superior to playing by feel or dealer rules, yet still a losing game long-term. \\

The implementation of High-Low Card Counting flipped the advantage entirely. The standard High-Low strategy yielded a positive return of 14,800 units, effectively creating a negative House Edge (Player Edge) of 1.48\%. When "Bet Spreads" were introduced—betting more when the count is high—the returns skyrocketed to 79,151.25 units with a Player Edge of 7.92\%. \\

The correlation between the "True Count" and the "Expected Return" provides the mathematical justification for these results. As seen in the generated scatter plots, both Basic Strategy and Card Counting show a positive slope (0.0041 and 0.0025-0.0518 respectively). This indicates that as the True Count rises, the expected return for a hand increases. The failure of Basic Strategy lies in the fact that it forces the player to bet during negative counts where the probability of winning is low. By identifying these high-probability moments via the True Count and increasing wager sizes (Bet Spreads), the simulation proves that the player can overcome the casino's inherent advantage.

\section{Further Implementations}
First, the current simulation runs sequentially on a single thread. Future iterations of the codebase will implement multithreading to parallelize the simulation across multiple CPU cores. By distributing the workload, the simulation volume can be increased from one million to hundreds of millions of hands in the same timeframe. According to the Law of Large Numbers, this massive increase in sample size would drastically reduce variance and narrow the confidence intervals of the results, providing much higher statistical confidence in the expected returns and risk of ruin calculations. \\

Second, the strategy engine can be upgraded to support playing deviations (often referred to as "indices"). The current model utilizes Basic Strategy for hand decisions and the True Count for betting sizing. However, advanced counters alter their playing decisions based on the count—for example, standing on a "Hard 16" against a Dealer 10 when the True Count is sufficiently high. Implementing these deviations would minimize losses in disadvantageous situations and maximize gains beyond simple bet spreading. \\

Finally, to mirror the adversarial nature of real-world casinos, the simulation must account for detection avoidance (camouflage). In a physical casino, perfect bet spreads and machine-like play attract scrutiny from pit bosses, leading to "back-offs." A more robust simulation would introduce variables for "cover play"—intentionally making sub-optimal bets or plays to blend in with recreational gamblers. This would allow for an analysis of the trade-off between theoretical maximum yield (EV) and practical longevity at the table.

\pagebreak

\newgeometry{top=0.5in, bottom=1.5in, left=0.5in, right=0.5in}

\begin{figure}
    \centering
    \includegraphics[width=0.65\linewidth]{DealerStrategy.png}
    \caption{Analysis of Dealer Strategy Performance}
    \label{fig:placeholder}
\end{figure}

\begin{figure}
    \centering
    \includegraphics[width=0.65\linewidth]{BasicStrategy.png}
    \caption{Analysis of Basic Strategy Performance}
    \label{fig:placeholder}
\end{figure}

\begin{figure}
    \centering
    \includegraphics[width=0.65\linewidth]{HighLowCardCounting.png}
    \caption{Analysis of High Low Card Counting + Basic Strategy Performance}
    \label{fig:placeholder}
\end{figure}

\begin{figure}
    \centering
    \includegraphics[width=0.65\linewidth]{BetSpreads.png}
    \caption{Analysis of High Low Card Counting + Bet Spreads + Basic Strategy Performance}
    \label{fig:placeholder}
\end{figure}

\restoregeometry
\clearpage

\section{Works Cited}

\begin{enumerate}
\item Blackjack Apprenticeship. “How to Play (and Win) at Blackjack: The Expert’s Guide.” YouTube, YouTube, 13 Nov. 2018, \url{www.youtube.com/watch?v=PljDuynF-j0}.

\item “How to Play Blackjack.” The Venetian Resort Las Vegas, \url{www.venetianlasvegas.com/casino/table-games/how-to-play-blackjack.html#:~:text=In%20Blackjack%2C%20everyone%20plays%20against,and%20the%20wager%20is%20lost}. Accessed 10 May 2024.

\item Shackleford, Michael. “Blackjack Basic Strategy.” Wizard of Odds, Wizard of Odds, 5 Oct. 2023, \url{wizardofodds.com/games/blackjack/strategy/calculator/}.

\item Shackleford, Michael. “Card Counting.” Wizard Of Odds > Guide to Gambling Games \& Online Casinos, Wizard of Odds, 21 Jan. 2019, \url{wizardofodds.com/games/blackjack/card-counting/introduction/}.

\item Shackleford, Michael. “Introduction to the High-Low Card Counting Strategy.” Wizard of Odds, Wizard of Odds, 27 Nov. 2023, \url{wizardofodds.com/games/blackjack/card-counting/high-low/}.

\item Younas, Aneeca. “Blackjack Odds Explained - House Edge and Payout.” Techopedia, \url{www.techopedia.com/gambling-guides/blackjack-odds-probability}. Accessed 11 May 2024.

\end{enumerate}
\end{document}